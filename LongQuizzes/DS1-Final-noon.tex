\documentclass[11pt, oneside]{article}   	% use "amsart" instead of "article" for AMSLaTeX format
\usepackage[margin=0.75in]{geometry}                		% See geometry.pdf to learn the layout options. There are lots.
\geometry{letterpaper}                   		% ... or a4paper or a5paper or ... 
%\geometry{landscape}                		% Activate for rotated page geometry
\usepackage[parfill]{parskip}    		% Activate to begin paragraphs with an empty line rather than an indent
\usepackage{graphicx}				% Use pdf, png, jpg, or eps§ with pdflatex; use eps in DVI mode
								% TeX will automatically convert eps --> pdf in pdflatex		
\usepackage{amssymb}
\usepackage{amsthm}

%SetFonts

%SetFonts

\usepackage{amsmath}
%\DeclareMathOperator{\mod}{mod}

\let\emptyset\varnothing

\newcommand{\reals}{\mathbb{R}}
\newcommand{\realsText}{$\mathbb{R}$}
\newcommand{\ints}{\mathbb{Z}}
\newcommand{\intsText}{$\mathbb{Z}$}
\renewcommand{\mod}{\ \mathrm{mod}\ }


\title{Final Exam}
\author{Discrete Structures 1}
\date{8 May 2023}							% Activate to display a given date or no date

\begin{document}
\maketitle
%\section{}
%\subsection{}
\begin{center}
Name: \_\_\_\_\_\_\_\_\_\_\_\_\_\_\_\_\_\_\_\_\_\_\_\_\_\_\_\_ \\(please write legibly) 
\end{center}

\begin{center}
\begin{tabular}{|c|l|l|}
\hline
\textbf{Question} & \multicolumn{1}{|c|}{\textbf{Topic}} & \multicolumn{1}{|c|}{\textbf{Value}}\\
\hline
\hline
1 & (functions) & \hspace{3em}/36\\ \hline
2 & (predicate logic) & \hspace{3em}/30\\ \hline
3 & (propositional logic) & \hspace{3em}/35\\ \hline
4 & (propositional logic) & \hspace{3em}/30\\ \hline
5 & (proofs) &  \hspace{3em}/15\\ \hline
6 & (proofs) & \hspace{3em}/30\\ \hline
7 & (proof by contradiction) & \hspace{3em}/30\\ \hline
8 & (proof by contrapositive) & \hspace{3em}/40\\ \hline
9 & (proof by cases) & \hspace{3em}/40\\ \hline
10 & (proofs) & \hspace{3em}/15\\ \hline
 & \textbf{bonus} &\\ \hline
 \hline
\hline
\multicolumn{2}{|l|}{\textbf{Total}} & \hspace{4em}/301\\
\hline
\end{tabular}
\end{center}


\textbf{\underline{There are a few rules:}}

(1) You are not allowed to use outside online resources. 
No outside help (e.g., from a classmate, a friend, online search, documents on your own laptop, etc.) will be tolerated. 
Any attempt to obtain help or information about the exam will be reported.

(2) \textbf{You are not allowed to use headphones during the exam.} Your phone should be in your backpack.

(3) You should only have a pencil (or pen), an eraser, and a pencil sharpener with you on the table at the time of the exam. 
Everything else should be safely packed in your backpack and not to be used at any time during the exam.

(4) You were told to attend to the bathroom before the exam starts: 
you will not be allowed to leave the room during the first hour of the exam (unless you have a doctor’s note to indicate otherwise).

\textbf{\underline{A few pieces of advice:}}

(1) Read the questions carefully and try the tracing exercises on draft paper before you answer on your exam copy.

(2) Pay careful attention to the instructions written in the exam.

\textbf{\underline{Please write legibly and in a structured manner:}} 
keep in mind that what you write needs to be read. Answers that are unreadable or hard to follow will not receive full credit.




\clearpage

\begin{enumerate}

\item 
Select the smallest size set that represents the domain and range of each of the following functions:
\begin{center}
\begin{tabular}{lcc}
Function & Domain & Range\\
\hline
\hline
\vspace{-.5em}\\
$f(x) = x \mod 3$ & \_\_\_\_\_\_\_\_\_ & \_\_\_\_\_\_\_\_\_ \vspace{.5em}\\
$g(x) = 3 \mod \left(\left(x \mod 2\right) + 4 \right)$ & \_\_\_\_\_\_\_\_\_ & \_\_\_\_\_\_\_\_\_ \vspace{.5em}\\
$h(x) = 3x^2+\pi x + \sqrt{2}$ & \_\_\_\_\_\_\_\_\_ & \_\_\_\_\_\_\_\_\_ \vspace{.5em}\\
$\ell(x) = \left|\left\lfloor x\right\rfloor\right|$ & \_\_\_\_\_\_\_\_\_ & \_\_\_\_\_\_\_\_\_ \vspace{.5em}\\
$e(x) = \left| x\right|$ & \_\_\_\_\_\_\_\_\_ & \_\_\_\_\_\_\_\_\_ \vspace{.5em}\\
$r(x) = x \mod 4 + \frac{3}{2}$ & \_\_\_\_\_\_\_\_\_ & \_\_\_\_\_\_\_\_\_ \vspace{.5em}\\
\end{tabular}
\begin{tabular}{ll}
(A) \intsText &  (H) $\{0\}$\\
(B) $\ints^{\ge1}$ & (K) $\{1\}$\\
(C) $\ints^{\ge0}$ & (M) $\{2\}$\\
(D) \realsText & (N) $\{3\}$\\
(E) $\reals^{\ge1}$ & (O) $\{0,1\}$\\
(F) $\reals^{\ge0}$ & (P) $\{0,1,2\}$\\
(G) $\emptyset$ & (Q) $\{0,1,2,3\}$\\ 
\end{tabular}
\end{center}

\item
For each of the following statements, write a qualified predicate logic statement. 
The variables will be given in parenthesis. 
\begin{enumerate}
\item Some student ($s\in S$) in the class has the highest grade ($g(s)$) in the class. 
(you will likely need to define a second student $t\in S$). \vspace{3em}
\item For all topics topics ($t\in T$) there is at least one questions ($q\in Q$) that at least one student ($s\in S$) will get correct ($C(t,q,s)=true$).\vspace{3em}
\end{enumerate}

\item \label{q:truthtable} Construct a truth table to evaluate the following expression:
\[
\left(p \wedge \neg q\right) \vee \left(\neg q \wedge \neg r\right)
\]
\vspace{18em}

\clearpage
\item Using the expression and truth table from Question~\ref{q:truthtable}, answer the following:
\begin{enumerate}
\item Is the expression a tautology? why or why not? \vspace{3em}
\item Is the expression satisfiable? Why or why not? \vspace{3em}
\item Is the expression in Disjunctive Normal Form, Conjunctive Normal Form, or neither? Explain your response. \vspace{3em}
\end{enumerate}

\item Disprove the following: \textit{A coin system with coins of value 4 and 7 cents can make any amount of change above 11 cents.}
\vspace{3em}


\item Prove the following claim: 
\begin{center}
\textit{There is a course taught in the CS department, that is one credit hour. }
\end{center}
\vspace{3em}

\item We want to prove the following using contradiction, \textbf{what is the initial assumption}?
\begin{center}
\textit{Any product $xy$ of two odd numbers $x$ and $y$ is odd.}
\end{center}
\vspace{3em}

\item If instead we wanted to prove the claim in the previous question by \textit{contrapositive}, 
\begin{enumerate}
\item assuming we represent the initial implication as $p \implies q$, what is $p$?
\vspace{2em}
\item what is $q$?
\vspace{2em}
\item what is $\neg q$?
\vspace{2em}
\item what is $\neg p$?
\vspace{2em}
\end{enumerate}

\clearpage
\item We want to prove the claim below:
\begin{center}
\textit{For all numbers $x\in\ints^{\ge1}$, if $x\mod 8 \in\{2,3,6,7\}$ \\
then the second to last bit in the binary representation is a 1.}
\end{center}
\begin{enumerate}
\item what would the different cases be in a proof by cases? \vspace{8em}
\item once we have proven each of the cases (no need to do it here, just assume we did), 
what is the final step in a proof by cases? (you should provide the final statement/argument.)\vspace{8em}
\end{enumerate}

\item Identify and describe the flaw in the following direct proof. 

\textbf{False Claim:} 1=0

\begin{proof}
Let $x, y \in \ints^{\ge1}$ where $x=y$.
\[
\begin{aligned}
x &= y & \text{by assumption}\\
x^2 &= x\cdot y & \text{multiply both sides by $x$}\\
x^2 - y^2 &= x\cdot y - y^2 & \text{subtract $y^2$ from both sides}\\ 
\left(x-y\right)\left(x+y\right) &= y \left(x-y\right) & \text{algebra}\\
x + y &= y & \text{divide both sides by $(x-y)$}\\
2 y &= y & \text{because $x=y$}\\
2 &= 1 & \text{divide by $y$, which is non-zero}\\
1 &= 0 & \text{subtract 1 from both sides}\\
\end{aligned}
\]
\end{proof}
\vspace{5em}

\end{enumerate}

\end{document} 