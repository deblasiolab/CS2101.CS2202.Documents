\documentclass[11pt, oneside]{article}   	% use "amsart" instead of "article" for AMSLaTeX format
\usepackage[margin=0.75in]{geometry}                		% See geometry.pdf to learn the layout options. There are lots.
\geometry{letterpaper}                   		% ... or a4paper or a5paper or ... 
%\geometry{landscape}                		% Activate for rotated page geometry
\usepackage[parfill]{parskip}    		% Activate to begin paragraphs with an empty line rather than an indent
\usepackage{graphicx}				% Use pdf, png, jpg, or eps§ with pdflatex; use eps in DVI mode
								% TeX will automatically convert eps --> pdf in pdflatex		
\usepackage{amssymb}
\usepackage{amsthm}

%SetFonts

%SetFonts

\usepackage{amsmath}
%\DeclareMathOperator{\mod}{mod}

\let\emptyset\varnothing

\newcommand{\reals}{\mathbb{R}}
\newcommand{\realsText}{$\mathbb{R}$}
\newcommand{\ints}{\mathbb{Z}}
\newcommand{\intsText}{$\mathbb{Z}$}
\renewcommand{\mod}{\ \mathrm{mod}\ }


\title{Final Exam}
\author{Discrete Structures 2}
\date{8 May 2023}							% Activate to display a given date or no date

\begin{document}
\maketitle
%\section{}
%\subsection{}
\begin{center}
Name: \_\_\_\_\_\_\_\_\_\_\_\_\_\_\_\_\_\_\_\_\_\_\_\_\_\_\_\_ \\(please write legibly) 
\end{center}

\begin{center}
\begin{tabular}{|c|l|l|}
\hline
\textbf{Question} & \multicolumn{1}{|c|}{\textbf{Topic}} & \multicolumn{1}{|c|}{\textbf{Value}}\\
\hline
\hline
1 & (functions) & \hspace{3em}/24\\ \hline
2 & (induction) & \hspace{3em}/25\\ \hline
3 & (induction) & \hspace{3em}/24\\ \hline
4 & (graphs) & \hspace{3em}/25\\ \hline
5 & (counting) &  \hspace{3em}/25\\ \hline
6 & (counting) & \hspace{3em}/25\\ \hline
7 & (graphs/counting) & \hspace{3em}/25\\ \hline
8 & (graphs) & \hspace{3em}/25\\ \hline
9 & (counting) & \hspace{3em}/25\\ \hline
10 & (asymptotics) & \hspace{3em}/30\\ \hline
11 & (proofs) & \hspace{3em}/25\\ \hline
 \hline
\hline
\multicolumn{2}{|l|}{\textbf{Total}} & \hspace{4em}/275\\
\hline
\end{tabular}
\end{center}


\textbf{\underline{There are a few rules:}}

(1) You are not allowed to use outside online resources. 
No outside help (e.g., from a classmate, a friend, online search, documents on your own laptop, etc.) will be tolerated. 
Any attempt to obtain help or information about the exam will be reported.

(2) \textbf{You are not allowed to use headphones during the exam.} Your phone should be in your backpack.

(3) You should only have a pencil (or pen), an eraser, and a pencil sharpener with you on the table at the time of the exam. 
Everything else should be safely packed in your backpack and not to be used at any time during the exam.

(4) You were told to attend to the bathroom before the exam starts: 
you will not be allowed to leave the room during the first hour of the exam (unless you have a doctor’s note to indicate otherwise).

\textbf{\underline{A few pieces of advice:}}

(1) Read the questions carefully and try the tracing exercises on draft paper before you answer on your exam copy.

(2) Pay careful attention to the instructions written in the exam.

\textbf{\underline{Please write legibly and in a structured manner:}} 
keep in mind that what you write needs to be read. Answers that are unreadable or hard to follow will not receive full credit.




\clearpage

\begin{enumerate}

\item 
Select the smallest size set that represents the domain and range of each of the following functions:
\begin{center}
\begin{tabular}{lcc}
Function & Domain & Range\\
\hline
\hline
\vspace{-.5em}\\
$f(x) = x \mod 3$ & \_\_\_\_\_\_\_\_\_ & \_\_\_\_\_\_\_\_\_ \vspace{.5em}\\
$g(x) = 3 \mod \left(\left(x \mod 2\right) + 4 \right)$ & \_\_\_\_\_\_\_\_\_ & \_\_\_\_\_\_\_\_\_ \vspace{.5em}\\
$h(x) = 3x^2+\pi x + \sqrt{2}$ & \_\_\_\_\_\_\_\_\_ & \_\_\_\_\_\_\_\_\_ \vspace{.5em}\\
$\ell(x) = \left|\left\lfloor x\right\rfloor\right|$ & \_\_\_\_\_\_\_\_\_ & \_\_\_\_\_\_\_\_\_ \vspace{.5em}\\
$e(x) = \left| x\right|$ & \_\_\_\_\_\_\_\_\_ & \_\_\_\_\_\_\_\_\_ \vspace{.5em}\\
$r(x) = x \mod 4 + \frac{3}{2}$ & \_\_\_\_\_\_\_\_\_ & \_\_\_\_\_\_\_\_\_ \vspace{.5em}\\
\end{tabular} \hspace{0em}
\begin{tabular}{ll}
(A) \intsText &  (H) $\{0\}$\\
(B) $\ints^{\ge1}$ & (K) $\{1\}$\\
(C) $\ints^{\ge0}$ & (M) $\{2\}$\\
(D) \realsText & (N) $\{3\}$\\
(E) $\reals^{\ge1}$ & (O) $\{0,1\}$\\
(F) $\reals^{\ge0}$ & (P) $\{0,1,2\}$\\
(G) $\emptyset$ & (Q) $\{0,1,2,3\}$\\ 
\end{tabular}
\end{center}

%%%%%%%%%%%%%%%%
\item Complete the proof below:
We will prove by \textit{weak} induction on $n$ that \[\sum_{i=1}^{n}f_i = f_{n+2}-1\] (where $f_k$ is the $k$-th fibonacci number). 

\textbf{Base cases} ($n=1$ and $n=2$):
 \[\sum_{i=1}^{1}f_i = f_1 = 1 = 2 - 1 = f_{3}-1\] and
 \[\sum_{i=1}^{2}f_i = f_1 + f_2 = 1 + 1 = 2 = 3 - 1 = f_{4}-1.\]
 
\textbf{Inductive case} ($n\ge 3$):\\
\hspace*{3em}We will assume the inductive hypothesis holds for $n-1$, that is $\sum_{i=1}^{n-1}f_i = f_{n+1}-1$. \\
\hspace*{3em} (note you will likely use direct prove to complete the rest of the inductive case)
 \vspace{16em}
 
 \clearpage
%%%%%%%%%%%%%%%%
\item We want to prove the following by some type of induction, answer the questions below. 
\begin{center}
\textit{
For $x\in\ints^{>0}$ that are powers of two,\\
the closed for solution of the recurrence relation $T(x) = 3T(\frac{n}{2}) + 4$
is $\Theta\left(3^{\log n}\right)$
}
\end{center}
\begin{enumerate}
\item What type of induction will we use? (i.e. ``weak'', strong, structural, etc) Why?\vspace{3em}
\item What value of $x$ will be the base case? \vspace{3em}
\item What is the \textit{inductive hypothesis}? (not the whole inductive step)\vspace{3em}
\end{enumerate}


%%%%%%%%%%%%%%%%
\item We know that a tree is a connected acyclic graph. 
Prove or disprove: 
\begin{center}
\textit{Any graph $G=\langle V,E\rangle$ with $|E|=|V|-1$ 
must be a tree. }
\end{center}
\vspace{5em}


%%%%%%%%%%%%%%%%
\item How many permutations are there of \texttt{SUMMER}? \vspace{5em}

%%%%%%%%%%%%%%%%
\item You're taking a classics class this semester that required you to buy $y$ books, 
but only $x$ of them fit in your backpack at once. 
What formula describes the number of combinations 
you must consider when choosing what to bring to campus on any given day? \vspace{5em}

%%%%%%%%%%%%%%%%
\item Thinking about the graphs quiz, assume we have a bipartite graph $G=\langle L\cup R, V\rangle$, 
where $|L|=|R|=|V|=n$, meaning each node has exactly one connection. 

How many different graphs are there that satisfy this condition? (Another way to think of it, given fixed $L$ and $R$ how many different sets $V$ are there such that $|V|=n$.)
\vspace{5em}

\clearpage
%%%%%%%%%%%%%%%%
\item Prove or disprove the following statement: 
\begin{center}
\textit{Given a tree $T=\langle V, E \rangle$, there must exist a root $r\in V$ which has at most two children. }
\end{center}
\vspace{7em}

%%%%%%%%%%%%%%%%
\item You're writing a method to find the pair of elements in an array of size $k$ 
that have the highest product.
How many products do you need to calculate 
(in the naive implication that just computes all pair's products)?
(Note, this answer should be an exact number, \textbf{not} in $O,\Theta,\Omega$ notation). 
\vspace{5em}

%%%%%%%%%%%%%%%%
\item You're working in a team to construct an program for an assignment. 
Your two colleagues made modules that you need to use and they reported the following to you: 
\begin{itemize}
\item Method A has a worst-case running time of $O(n)$
\item Method B has a worst-case running time of $\Theta(\log n)$
\end{itemize}
You wrote a method that runs Method A $\frac{n}{2}$~times on inputs of size~$2\cdot n$, 
then runs Method B 10~times on inputs of size $n$. 
What can you conclude about the worst-case running time of your program as a whole? 
\vspace{8em}

%%%%%%%%%%%%%%%%
\item Identify and describe the flaw in the following direct proof. 

\textbf{False Claim:} 1=0

\begin{proof}
Let $x, y \in \ints^{\ge1}$ where $x=y$.
\[
\begin{aligned}
x &= y & \text{by assumption}\\
x^2 &= x\cdot y & \text{multiply both sides by $x$}\\
x^2 – y^2 &= x\cdot y – y^2 & \text{subtract $y^2$ from both sides}\\ 
\left(x-y\right)\left(x+y\right) &= y \left(x-y\right) & \text{algebra}\\
x + y &= y & \text{divide both sides by $(x-y)$}\\
2 y &= y & \text{because $x=y$}\\
2 &= 1 & \text{divide by $y$, which is non-zero}\\
1 &= 0 & \text{subtract 1 from both sides}\\
\end{aligned}
\]
\end{proof}
\vspace{5em}

\end{enumerate}

\end{document}
