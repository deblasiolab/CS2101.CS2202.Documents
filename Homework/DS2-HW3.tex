\documentclass[11pt, oneside]{article}   	% use "amsart" instead of "article" for AMSLaTeX format
\usepackage[margin=1in]{geometry}                		% See geometry.pdf to learn the layout options. There are lots.
\geometry{letterpaper}                   		% ... or a4paper or a5paper or ... 
%\geometry{landscape}                		% Activate for rotated page geometry
%\usepackage[parfill]{parskip}    		% Activate to begin paragraphs with an empty line rather than an indent
\usepackage{graphicx}				% Use pdf, png, jpg, or eps§ with pdflatex; use eps in DVI mode
								% TeX will automatically convert eps --> pdf in pdflatex		
\usepackage{amssymb}
\usepackage{awesomebox}
%SetFonts

%SetFonts

\usepackage{amsmath}
%\DeclareMathOperator{\mod}{mod}
\let\emptyset\varnothing

\newcommand{\reals}{\mathbb{R}}
\newcommand{\realsText}{$\mathbb{R}$}
\newcommand{\ints}{\mathbb{Z}}
\newcommand{\intsText}{$\mathbb{Z}$}

\title{Homework 3}
\author{Discrete Structures 2}
\date{due: 28 February 2023, 8:00am}							% Activate to display a given date or no date

\begin{document}
\maketitle
%\section{}
%\subsection{}

Your task for this homework will be to answer the following questions without using any calculating resources. 
Your responses should be submitted via blackboard by the due date above as a PDF (submissions in any other format will be returned to the user and a resubmissions will be requested). 
You are free to use whatever tools you would like to generate the response document: 
scanned hand-written paper, 
tablet generated hand-written, 
microsoft word (with this option, please use the equation editor to correctly format your responses), 
\LaTeX, etc.
Your TA, IA, and Instructor are available to help during their designated office hours or via email 
(note that emails sent during non-business hours may not be responded to until the next working day). 

\importantbox{
\textbf{Note:} all of these questions are on topics from chapters 5; thus you will only be proving by induction in this homework assignment. 
}
\begin{enumerate}
% 5.2
\item Prove the following claim holds $\forall n \in \ints^{\ge0}$ by induction on $n$:
\[
\sum_{i=0}^n i^3 = \frac{n^4 + 2n^3 + n^2}{4}
\]

%5.33
\item Prove Bernoulli’s inequality by induction:  For an arbitrary $x\in\reals$, such that $x\ge-1$, 
prove by induction on $n$ that
\[
(1+x)^n \ge1+nx
\]
for any positive integer $n$.

%5.41
\item Prove by induction on n that $8^n - 3^n$ is divisible by 5 for any nonnegative integer.
\tipbox{This seems like it may use modulus somewhere in the proof, it does not.}

%5.61
\item Prove Cassini’s identity: $f_{n-1} \cdot f_{n+1} - (f_n)^2 = (-1)^n$ for any $n\ge2$, there $f_n$ is the $n$-th Fibonacci number. 
\notebox{remember, the $n$-th Fibonacci number $f_n = f_{n-1} + f_{n-2}$ for all $n\ge 3$ and $f_1 = f_2 = 1$.}

\clearpage
%5.56
\item Consider a sport in which teams can score two types of goals, 
worth either 3 points or 7 points.
For example, Team Miners might (theoretically speaking) score 32 points by accumulating, in succession, 3, 7, 3, 7, 3, 3, 3, and 3 points. 
Find the smallest possible $n_0$ such that, for any $n\ge n_0$, a team can score exactly $n$ points in a game. 
Prove your answer correct by strong induction.
\notebox{This means you need to first figure out $n_0$, then prove that the claim is true for all larger $n$.}
\tipbox{You may need to employ more than one base case here}
\end{enumerate}

\end{document}