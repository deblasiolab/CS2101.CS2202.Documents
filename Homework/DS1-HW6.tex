\documentclass[11pt, oneside]{article}   	% use "amsart" instead of "article" for AMSLaTeX format
\usepackage[margin=1in]{geometry}                		% See geometry.pdf to learn the layout options. There are lots.
\geometry{letterpaper}                   		% ... or a4paper or a5paper or ... 
%\geometry{landscape}                		% Activate for rotated page geometry
%\usepackage[parfill]{parskip}    		% Activate to begin paragraphs with an empty line rather than an indent
\usepackage{graphicx}				% Use pdf, png, jpg, or eps§ with pdflatex; use eps in DVI mode
								% TeX will automatically convert eps --> pdf in pdflatex		
\usepackage{amssymb}
\usepackage{awesomebox}
%SetFonts

%SetFonts

\usepackage{amsmath}
\DeclareMathOperator{\plainmod}{\text{ mod }}
\let\emptyset\varnothing

\newcommand{\reals}{\mathbb{R}}
\newcommand{\realsText}{$\mathbb{R}$}
\newcommand{\ints}{\mathbb{Z}}
\newcommand{\intsText}{$\mathbb{Z}$}

\title{Homework 6}
\author{Discrete Structures 1}
\date{due: 20 April 2023, 8:00am}							% Activate to display a given date or no date

\begin{document}
\maketitle
%\section{}
%\subsection{}

Your task for this homework will be to answer the following questions without using any calculating resources. 
Your responses should be submitted via blackboard by the due date above as a PDF (submissions in any other format will be returned to the user and a resubmissions will be requested). 
You are free to use whatever tools you would like to generate the response document: 
scanned hand-written paper, 
tablet generated hand-written, 
microsoft word (with this option, please use the equation editor to correctly format your responses), 
\LaTeX, etc.
Your TA, IA, and Instructor are available to help during their designated office hours or via email 
(note that emails sent during non-business hours may not be responded to until the next working day). 

%\importantbox{
%\textbf{Note:} all of these questions are on topics from chapters 5; thus you will only be proving by induction in this homework assignment. 
%}
\begin{enumerate}
\item If you wanted to prove the following claim \emph{by cases}, what would the cases be that you're going to divide the problem into:
\begin{center}
\textit{\textbf{claim:} For any number $x\in\ints$, $x^2$ is even.}
\end{center}

\item Prove that the binary representation of any odd integer ends with a 1.

\item We want a \emph{proof by contrapositive} for the following claim, what is the contrapositive of the statement:
\begin{center}
\textit{\textbf{claim:} Let $n\in\ints^{\ge0}$. If $n\mod4\in\{2,3\}$, then $n$ is not a perfect square.}
\end{center}

\item If you were to prove the following claim using \emph{proof by contradiction}, what would be the initial assumption:
\begin{center}
\textit{\textbf{claim:} Let $x, y \in\reals$.  If $x^2-y^2 = 1$, then $x$ or $y$ (or both) is not an integer.}
\end{center}

\item Disprove the following \emph{by counterexample}: \textit{If $x-y$ is rational, then $x$ and $y$ are rational.}

\item Prove that the area of a right triangle with legs $x$ and $y$ is $xy/2$.

\end{enumerate}
\end{document}