\documentclass[11pt, oneside]{article}   	% use "amsart" instead of "article" for AMSLaTeX format
\usepackage[margin=1in]{geometry}                		% See geometry.pdf to learn the layout options. There are lots.
\geometry{letterpaper}                   		% ... or a4paper or a5paper or ... 
%\geometry{landscape}                		% Activate for rotated page geometry
%\usepackage[parfill]{parskip}    		% Activate to begin paragraphs with an empty line rather than an indent
\usepackage{graphicx}				% Use pdf, png, jpg, or eps§ with pdflatex; use eps in DVI mode
								% TeX will automatically convert eps --> pdf in pdflatex		
\usepackage{amssymb}
\usepackage{awesomebox}
%SetFonts

%SetFonts

\usepackage{amsmath}
\DeclareMathOperator{\plainmod}{\text{ mod }}
\let\emptyset\varnothing

\newcommand{\reals}{\mathbb{R}}
\newcommand{\realsText}{$\mathbb{R}$}
\newcommand{\ints}{\mathbb{Z}}
\newcommand{\intsText}{$\mathbb{Z}$}

\title{Homework 4}
\author{Discrete Structures 1}
\date{due: 30 March 2023, 8:00am}							% Activate to display a given date or no date

\begin{document}
\maketitle
%\section{}
%\subsection{}

Your task for this homework will be to answer the following questions without using any calculating resources. 
Your responses should be submitted via blackboard by the due date above as a PDF (submissions in any other format will be returned to the user and a resubmissions will be requested). 
You are free to use whatever tools you would like to generate the response document: 
scanned hand-written paper, 
tablet generated hand-written, 
microsoft word (with this option, please use the equation editor to correctly format your responses), 
\LaTeX, etc.
Your TA, IA, and Instructor are available to help during their designated office hours or via email 
(note that emails sent during non-business hours may not be responded to until the next working day). 

%\importantbox{
%\textbf{Note:} all of these questions are on topics from chapters 5; thus you will only be proving by induction in this homework assignment. 
%}
\begin{enumerate}

\item True or False: it is always the case that
\[
a\mod 10 = (a+100)\plainmod 10 = (a-10)\plainmod 100
\]
for some $a\ge 20$. Justify your answer. 

\item What is 
\[
\left|\left\{x\plainmod y : x \in \ints^{\ge 0}\right\}\right|
\]
for a fixed, but arbitrary, number $y\in\ints^{\ge1}$? 
(note this is the set size, not the set itself.)

\item Reduce the following to a polynomial on $x$ (that is an equation that involves only x):
\[
\sum_{c=3}^5 \sum_{d=1}^c x^d
\]

%2.145
\item Supposed $A\times B = \{\langle1,1\rangle, \langle2,1\rangle\}$. What are the sets $A$ and $B$? \\(note that $\times$ is not multiplication, its cartesian product)

%2.174
\item For the vector $c=\langle 4 , 0 \rangle$, what is the value of $c\cdot c$? \\(note that $\cdot$ is not multiplication, its dot product)

%2.233-2.242
\item State the \textbf{domain} and \textbf{range} of the following functions, be as precise as possible (that means if you can place a restriction on one of our known sets, do so; i.e. $\mathbb{Z}^{\ge1}$ rather than just $\mathbb{Z}$).
\begin{enumerate}
\item $f(x) = |x|$
\item $f(x) = \lfloor x \rfloor$
\item $f(x) = x \mod 2$
\item $f(x) = 2 \mod x$
\end{enumerate}

%3.78-3.80
\item Show (using a truth table) that the following equivalences hold: 
\begin{enumerate}
\item $p \vee (q \vee r) \equiv (p \vee q) \vee r$
\item $p \wedge (q \wedge r) \equiv (p \wedge q) \wedge r$
\item $p \oplus (q \oplus r) \equiv (p \oplus q) \oplus r$
\end{enumerate}
\end{enumerate}
\end{document}