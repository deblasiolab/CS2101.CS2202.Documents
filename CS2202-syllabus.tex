%% adapted from Chris Bourke's syllabus template for CS 1
%% https://github.com/cbourke/ComputerScienceI
%% Accessed on 29 July 2020
%% Used and distributed under the CC BY-SA 4.0 License

\documentclass[12pt]{scrartcl}
%\usepackage{tagpdf}

\usepackage{epsfig,amssymb}

%\usepackage{draftwatermark}
%\SetWatermarkScale{7}

\usepackage{xcolor}
\usepackage{graphicx}
\usepackage{epstopdf}
\usepackage{multirow}
\usepackage{colortbl} 
\usepackage{xspace}
\usepackage[normalem]{ulem}
\usepackage{multicol}
\usepackage{todonotes}

\usepackage{tcolorbox}

\definecolor{steelblue}{RGB}{70, 130, 180}
\definecolor{darkred}{rgb}{0.5,0,0}
\definecolor{darkgreen}{rgb}{0,0.5,0}
\usepackage[pdflang={en-US}]{hyperref}
\hypersetup{
  letterpaper,
  colorlinks,
  linkcolor=darkgreen,
  citecolor=darkgreen,
  menucolor=darkred,
  urlcolor=blue,
  pdfpagemode=none,
  pdftitle={Syllabus},
  pdfauthor={Dan DeBlasio},
  pdfkeywords={}
}
\setcounter{tocdepth}{2}

\usepackage{fullpage}
\pagestyle{empty} %
\usepackage{subfigure}
\usepackage{enumitem}
\setenumerate{nolistsep}
\setitemize{nolistsep}
\renewcommand{\labelenumii}{\alph{enumii}.}


\setlength{\parindent}{0pt} %
\setlength{\parskip}{.25cm}
\usepackage{lastpage}
\usepackage{fancyhdr}
\renewcommand*{\titlepagestyle}{fancy}
\pagestyle{fancy}
\renewcommand{\headrulewidth}{0.0pt}
\renewcommand{\footrulewidth}{0.4pt}

\lhead{~}
\chead{~}
\rhead{~}
\lfoot{\Title ---Syllabus}
\cfoot{~}
\rfoot{\thepage\ / \pageref*{LastPage}}

\makeatletter
\title{Discrete Structures 2}\let\Title\@title
\subtitle{
{\small
Dr. Dan DeBlasio\\
Department of Computer Science\\
University of Texas at El Paso\\
}
\vskip-1cm}
\date{\small CS 2102 -- Spring 2023\\ \vspace{1em}Revised: \today\\(change since start of classes marked \change{}{in orange})}
\makeatother

\newcommand{\change}[2]{\textcolor{orange}{#2}}
%\newcommand{\change}[2]{#2}


%\tagpdfifpdftexT
% {
%  \pdfcatalog{/Lang (en-US)}
%  \usepackage[T1]{fontenc}
% }
% 
%\tagpdfsetup{activate-all,tabsorder=structure}

\begin{document}
%\tagstructbegin{tag=Document}

\maketitle
%
%\begin{center}
%{\Huge\color{red}DRAFT}
%\end{center}
%
%%%%%%%%%%%%%%%%%%%%%%%%%%%%%%%%%%%%
%%%%%%%%%%%%%%%%%%%%%%%%%%%%%%%%%%%%
%\section{General Information}
%%%%%%%%%%%%%%%%%%%%%%%%%%%%%%%%%%%%
%%%%%%%%%%%%%%%%%%%%%%%%%%%%%%%%%%%%
%\tagmcbegin{tag=P}
\paragraph{Course Objectives:} Upon successful completion of the course, you will have a solid understanding of the following topics: propositional logic, proofs, sets, functions, and relations, counting, induction and recursion. Most importantly, you will know of and be able to articulate how these topics are essential in Computer Science.
%\tagmcend

\paragraph{Prerequisite:} Students must have passed both of the following with a C or better:
\begin{itemize}
\item MATH 1411
\item CS 2101
\end{itemize}

\paragraph{Knowledge and Abilities Required Before Entering the Course:} Students are assumed to be comfortable with basic algebra, and have some exposure to programatic thinking and problem solving. 

\paragraph{Topics covered this semester:}
\begin{itemize} 
\item counting methods and how their implications in computer science
\item graphs and trees as mathematical concepts
\item reinforcing proof methods from discrete structures 1  
\end{itemize}

\clearpage
\tableofcontents

%%%%%%%%%%%%%%%%%%%%%%%%%%%%%%%%%%%%
\section{Logistics}
%%%%%%%%%%%%%%%%%%%%%%%%%%%%%%%%%%%%
\paragraph{Synchronous course session times and locations:}
\begin{itemize}
\item TR 9:00am-10:20a --- PSCI 115
\end{itemize}

\paragraph{Textbook:} \textit{Connecting Discrete Mathematics and Computer Science, 2nd edition} by David Liben-Nowell from Cambridge University Press. 
Available at the bookstore and other booksellers (ISBN: 978-1-00-915049-1). 
A preprint (with some corrections missing) of the textbook is available online at \url{https://cs.carleton.edu/faculty/dln/book/} if you have ordered the book but are waiting for it to arrive.

This course, in combination with Discrete Structures 1, will cover the majority of this entire text. Within Discrete Structures 2 we will cover the contents of Chapters 6--9 and 10. 
Because previous instances of this course used a different book, we may at points go back to previous chapters in the book for a refresher on terminology.  

\paragraph{Communication platforms:}
\begin{itemize}
\item \textbf{Blackboard}  -- Used for announcements and grade tracking. All official grades and feedback will be sent through Blackboard. Students should monitor this site for important class information. 
\item \textbf{MS Teams} -- Used for office hours and intra-class discussions. Several channels will be available in the team for asking and answering questions, the instructional staff will answer questions posted on teams, but other students are encouraged to provide feedback as well. 
\end{itemize}

%%%%%%%%%%%%%%%%%%%%%%%%%%%%%%%%%%%%
\section{Instructional Staff}
%%%%%%%%%%%%%%%%%%%%%%%%%%%%%%%%%%%%

\begin{tabular}{lrl}
\multicolumn{3}{l}{\fontfamily{cmss}\selectfont \Large \textbf{Instructors}}\vspace{0.75em}\\
\textbf{Dr. Dan DeBlasio}  
 & email: & dfdeblasio@utep.edu\\
 & MS Teams: &  \href{http://teamsChat.deblasiolab.org}{\texttt{teamsChat.deblasiolab.org}} (direct message)\\
 & office: & CCSB 3.1008\\
& office hours:& T \& R 10:30a-11:30a \\
%& & [or ``Office Hours'' on the class team]\\
& appointments: & \href{http://calendly.deblasiolab.org}{\texttt{calendly.deblasiolab.org}}\\

\\
\multicolumn{3}{l}{\fontfamily{cmss}\selectfont \Large \textbf{Teaching Assistants}}\vspace{0.75em}\\
%
\textbf{Ivan Montoya Sanchez}
 & email: & iamontoyasa@miners.utep.edu\\
 & office hours:& \change{}{Mondays \& Wednesdays 1:30p-2:30p}\\
 & & \change{}{Tuesday 4:00pm-6:00pm}\\
 && \change{}{CCSB 1.0706 (TA Room)}
\\
\multicolumn{3}{l}{\fontfamily{cmss}\selectfont \Large \textbf{Instructional Assistants}}\vspace{0.75em}\\
%
\textbf{Alexis Rodriguez}
 & email: & arodriguez248@miners.utep.edu\\
 & office hours:& \change{}{Monday 3:00pm-5:00pm}\\
 & & \change{}{Thursday 1:00pm-3:00pm}\\
 && \change{}{CCSB G.0512 (Dungeon)}
%
%
\end{tabular}


%%%%%%%%%%%%%%%%%%%%%%%%%%%%%%%%%%%%
\section{Expectations}
%%%%%%%%%%%%%%%%%%%%%%%%%%%%%%%%%%%%

\paragraph{Communication:} Students are expected to consult their emails and blackboard messages \textit{at least} twice a week, and to answer these as relevant. 

\paragraph{Class and Lab Participation:} Keeping up with class contents and participation in both lecture and lab sessions are critical factors of your success in this course. 

\textit{Students should be on time for all scheduled sessions and attend the entire session.} 
%Attendance will be taken at every synchronous class and lab session and will count towards your class participation grade. 
This semester attendance will not be taken to discourage those who may be sick to stay home, 
but those students who attend and readily participate in the course do better on assessments.

Students should notify the instructor prior to missing a session if at all possible, and certainly right after if earlier was not possible. 
%The instructor will allow two unexcused absences per semester before having the option to deduct points from the final grade (5 points per subsequent unexcused absence). 
Students should submit their work on time and meet all deadlines. Failing to do so will affect the participation grade.


\textit{It is the student's responsibility to review the content covered during missed class(es) or labs, as well as the assignments given during their absence.}
Participation points also include completing post-lecture and post-labs online quizzes (when requested) that are administered as surveys to monitor students’ overall progress and potential struggles.

\textit{Students should be on task.} 
When in synchronous class or lab session, students are expected to direct their attention to the task / activity as directed by the lecture / lab instructor. 
For instance, synchronous class sessions are certainly not places for social-networking, working on homework, participating in other courses.

\textit{Professionalism:} 
Students should be professional in their communications, as the context permits.
Emails should contains subjects, the recipients should be addressed (i.e. ``Hello Dr. DeBlasio, ...''), and the email should be signed with your name. 
Real-time online communication (i.e. MS Teams), while less formal, should still be professional. 


%%%%%%%%%%%%%%%%%%%%%%%%%%%%%%%%%%%%
%%%%%%%%%%%%%%%%%%%%%%%%%%%%%%%%%%%%
\section{Grading}
%%%%%%%%%%%%%%%%%%%%%%%%%%%%%%%%%%%%
%%%%%%%%%%%%%%%%%%%%%%%%%%%%%%%%%%%%

Grades are communicated to students in a timely manner. 
It is the students’ responsibility to keep track of their grades by compiling the grades they receive. 
Your semester grade will be based on a combination of homework assignments, weekly quizzes, class participation,  mid-term assessment, and a final exam. 

The approximate percentages are as follows:
\begin{center}
\begin{tabular}{rl}
\textbf{20\% } & Class participation \\
\textbf{35\% } & Homework/Quizzes/In-class assignments grade\\
\textbf{25\% } & Mid-term assessment\\
\textbf{20\% } & Final exam\\
\end{tabular}
\end{center}
The base percentage-score-to-letter-grade conversion for this course is as follows: 

\begin{center}
\begin{tabular}{rl}
\textbf{90\%}& or higher is guaranteed an A \\
\textbf{80\%}& or higher is guaranteed a B \\
\textbf{70\%}& or higher is guaranteed a C \\
\textbf{60\%}& or higher is guaranteed a D \\
\textbf{}& all lower grades are an F 
\end{tabular}
\end{center}
These minimums may be lowered without notice but will not be raised. 

%%%%%%%%%%%%%%%%%%%%%%%%%%%%%%%%%%%%
\subsection{Homework / Quizzes / In-class assignments}
%%%%%%%%%%%%%%%%%%%%%%%%%%%%%%%%%%%%

\subsubsection{Daily Quizzes}
The purpose of each quiz is to ensure that students are staying current with the weekly content and to verify that they have acquired the skills developed in class. 
Short quizzes are unannounced. 
All quizzes will be in person, 
%on blackboard, 
Quizzes are graded on submission not on correctness as a way to judge the progress of the student body, please try your best but note when you're making an educated guess. 

There will be no make-up on missed quizzes, but up to 3 quizzes can be dropped (missed) though the course of the semester.
 
 \subsubsection{Long Quizzes}
 Though the course of the semester we will have longer (graded) quizzes. 
 The modality of the quizzes will either be in class or online (to be determined). 
 They will be scheduled directly proceeding (the meeting before) each mid-term assessment and will cover similar material. 
 Details on these long quizzes will be announced in class as they occur. 
 
\subsubsection{In-class assignments}
There will be unannounced in-class assignments, to be turned in either by the end of the class or within a short period of time after the class (details will be given for each assignment). 
There will be no make-up for missed in-class assignments. 
Grades of such assignments will weigh equally with grades from quizzes. 


%%%%%%%%%%%%%%%%%%%%%%%%%%%%%%%%%%%%
\subsection{Exams \& Assessments}
%%%%%%%%%%%%%%%%%%%%%%%%%%%%%%%%%%%%
If you have test-taking difficulties in general, or if you have difficulties with our tests in particular, please request appropriate accommodation from UTEP’s Center for Accommodation and Students’ Services.

The purpose of the midterm assessment is to allow you to demonstrate mastery of course concepts covered thus far during the semester. 
Mid-term assessments will take place 
in class 
%on backboard with clear time constraints 
and is tentatively scheduled to be held in week 8. 
Make-up exams will be given only in extremely unusual circumstances. 
If you must miss an exam, please meet with an instructor, \textit{BEFORE} the exam. 

{\color{red}Students are required to be on time for assessments, a 10\% reduction to an assessment grade will be made if a student arrives more than 5 minutes after the start of the test, and 20\% at 10 minutes,.
A 100\% deduction will be made after 15 minutes.}
Arriving to a test late is disruptive to all students in the class and cannot be tolerated. 
 
%\paragraph{Extra-credit opportunity:} 
%for those students with (almost) perfect on-time homework (you can miss 1 deadline), a perfect grade of 100 will be added to the average making up the midterm exams grade.

\textbf{\underline{The final exam will be comprehensive.}}
You must take the final exam during the time shown in the schedule for the lecture section that you normally attend. 
Do not "drop in" to another section: there will not be a copy of the exam for you. 
This is University policy. If you have a scheduling conflict (e.g., if you are taking a final at EPCC) or if you are scheduled for three final exams in one day, see your instructor at least a week in advance for arranging accommodation.
\begin{tcolorbox}[colback=green!5,colframe=green!75!black,title=Final Exam Date]
\begin{center}
\textbf{Tuesday, May 9th, 2023; 10:00 am – 12:45 pm}.
 \end{center}
 \end{tcolorbox}


%%%%%%%%%%%%%%%%%%%%%%%%%%%%%%%%%%%%
%%%%%%%%%%%%%%%%%%%%%%%%%%%%%%%%%%%%
\section{Standing in the course}
%%%%%%%%%%%%%%%%%%%%%%%%%%%%%%%%%%%%
%%%%%%%%%%%%%%%%%%%%%%%%%%%%%%%%%%%%

%%%%%%%%%%%%%%%%%%%%%%%%%%%%%%%%%%%%
\paragraph{Standing in the Course:} 
%%%%%%%%%%%%%%%%%%%%%%%%%%%%%%%%%%%%

Students will have access to their grades for all assignments so that they can self-monitor their standing and progress. 
However, it is also completely fine for any student to come and talk to their instructor about their standing and work together to make sure the student is as successful as can be.

%%%%%%%%%%%%%%%%%%%%%%%%%%%%%%%%%%%%
\paragraph{Dropping the Course:} 
%%%%%%%%%%%%%%%%%%%%%%%%%%%%%%%%%%%%
Every semester, some students drop the course. We, instructors, completely understand and respect that. We only hereby ask students to inform us, ideally before, but in the worst-case right after, of their intention to drop the course. This is really important for us as it possibly informs us of ways in which to better serve our students.


%%%%%%%%%%%%%%%%%%%%%%%%%%%%%%%%%%%%
%%%%%%%%%%%%%%%%%%%%%%%%%%%%%%%%%%%%
\section{Special notices for COVID-19}
%%%%%%%%%%%%%%%%%%%%%%%%%%%%%%%%%%%%
%%%%%%%%%%%%%%%%%%%%%%%%%%%%%%%%%%%%

Please stay home if you have been diagnosed with COVID-19 or are experiencing COVID-19 symptoms. 
If you are feeling unwell, please let me know as soon as possible, so that we can work on appropriate accommodations. 
If you have tested positive for COVID-19, 
you are encouraged to report your results to covidaction@utep.edu, 
so that the Dean of Students Office can provide you with support and help with communication with your professors. 
The Student Health Center is equipped to provide COVID-19 testing. 
 
The Center for Disease Control and Prevention recommends that people 
in areas of substantial or high COVID-19 transmission wear face masks when indoors in groups of people. 
The best way that Miners can take care of Miners is to get the vaccine. 
If you still need the vaccine, it is widely available in the El Paso area, 
and will be available at no charge on campus during the first week of classes. 
For more information about the current rates, testing, and vaccinations, please visit epstrong.org.

\begin{tcolorbox}[colback=red!5,colframe=red!75!black,title=Masks in the classroom]
The instructional staff will be following CDC guidance on masking in groups based on the current transmission rates in El Paso Country\footnote{\url{https://www.cdc.gov/coronavirus/2019-ncov/your-health/covid-by-county.html}}. 
Please respect the choices of other who may prefer to be more cautious than recommendations. 
\end{tcolorbox}

%While there is not a plan to hold any meetings of 2401 on campus this semester, 
%as the university updates it's campus operations there may be situations that lead a student to be on campus. 
%The following are a summary of the universities policies regarding COVID-19.
%
%You must STAY OFF CAMPUS and REPORT if you:
%(1) have been diagnosed with COVID- 19, 
%(2) are experiencing COVID-19 symptoms, or 
%(3) have had recent contact with a person who has received a positive coronavirus test. 
%Reports should be made at \href{http://screening.utep.edu}{\texttt{screening.utep.edu}}. 
%If you know anyone who should report any of these three criteria, encourage them to report. 
%If the individual cannot report, you can report on their behalf by sending an email to \url{COVIDaction@utep.edu}.
%
%For each day that you attend campus—for any reason—you must complete the questions on the UTEP screening website (\href{http://screening.utep.edu}{\texttt{screening.utep.edu}}) prior to arriving on campus. 
%The website will verify if you are permitted to come to campus. 
%Under no circumstances should anyone come to class when feeling ill or exhibiting any of the known COVID-19 symptoms. 
%If you are feeling unwell, please let me know as soon as possible, 
%and alternative instruction will be provided. Students are advised to minimize the number of encounters with others to avoid infection.
%
%Wear face coverings when in common areas of campus or when others are present. 
%You must wear a face covering over your nose and mouth at all times in this class. 
%If you choose not to wear a face covering, you may not enter the classroom. 
%If you remove your face covering, you will be asked to put it on or leave the classroom. 
%Students who refuse to wear a face covering and follow preventive COVID-19 guidelines will be dismissed from the class and will be subject to disciplinary action according to Section 1.2.3 Health and Safety and Section 1.2.2.5 Disruptions in the UTEP Handbook of Operating Procedures.


%%%%%%%%%%%%%%%%%%%%%%%%%%%%%%%%%%%%
%%%%%%%%%%%%%%%%%%%%%%%%%%%%%%%%%%%%
\section{Resources}
%%%%%%%%%%%%%%%%%%%%%%%%%%%%%%%%%%%%
%%%%%%%%%%%%%%%%%%%%%%%%%%%%%%%%%%%%

%%%%%%%%%%%%%%%%%%%%%%%%%%%%%%%%%%%%
\paragraph{Special Accommodations: }
%%%%%%%%%%%%%%%%%%%%%%%%%%%%%%%%%%%%
If you have a disability and need classroom accommodations, please contact the Center for Accommodations and Support Services (CASS) at 747-5148 or by email to cass@utep.edu, or visit their office located in UTEP Union East, Room 106. For additional information, please visit the CASS website at \href{http://www.sa.utep.edu/cass}{\texttt{www.sa.utep.edu/cass}}. CASS’ staff are the only individuals who can validate and if need be, authorize accommodations for students with disabilities.


%%%%%%%%%%%%%%%%%%%%%%%%%%%%%%%%%%%%
\paragraph{Scholastic Dishonesty: }
%%%%%%%%%%%%%%%%%%%%%%%%%%%%%%%%%%%%
Any student who commits an act of scholastic dishonesty is subject to discipline. Scholastic dishonesty includes, but not limited to cheating, plagiarism, collusion, and submission for credit of any work or materials that are attributable to another person.

Cheating is:
\begin{itemize}
\item Copying from the test paper of another student
\item Communicating with another student during a test to be taken individually
\item Giving or seeking aid from another student during a test to be taken individually
\item Possession and/or use of unauthorized materials during tests (i.e. crib notes, class notes, books, etc.)
\item Substituting for another person to take a test
\item Falsifying research data, reports, academic work offered for credit
\end{itemize}

Plagiarism is:
\begin{itemize}
\item Using someone’s work in your assignments without the proper citations
\item Submitting the same paper or assignment from a different course, without direct permission of instructors
\item[]\vspace{1em} To avoid plagiarism, see: \\{\footnotesize\url{https://www.utep.edu/student-affairs/osccr/_Files/docs/Avoiding-Plagiarism.pdf}}
\end{itemize}
                               
Collusion is:
\begin{itemize}
\item Unauthorized collaboration with another person in preparing academic assignments
\end{itemize}

\begin{tcolorbox}[colback=red!5,colframe=red!75!black,title=Important!]
When in doubt on any of the above, please contact your instructor to check if you are following authorized procedure. Also, please check the UTEP’s Handbook of Operating Procedures at: hoop.utep.edu. 
\end{tcolorbox}



%%%%%%%%%%%%%%%%%%%%%%%%%%%%%%%%%%%%
%%%%%%%%%%%%%%%%%%%%%%%%%%%%%%%%%%%%
\section{Detailed Learning Outcomes}
%%%%%%%%%%%%%%%%%%%%%%%%%%%%%%%%%%%%
%%%%%%%%%%%%%%%%%%%%%%%%%%%%%%%%%%%%

\subsection*{Level 3: Synthesis and Evaluation}
Level 3 outcomes are those in which the student can apply the material in new situations. This is the highest level of mastery. On successful completion of this course, students will be able to identify, implement and use the following data structures as appropriate for a given problem:
\begin{enumerate}
\item \textbf{Reason about the complexity of algorithms using counting techniques and properties of graphs}
\item \textbf{Model computer science problems using graphs and trees}
\item \textbf{Lay out a proof plan for existential and universal proofs, be able to identify shortcomings of some types of
proving strategies}
\item \textbf{Identify an inductive structure of a set: use it to conduct an inductive proof and to set a recurrence relation.}
\end{enumerate}

\subsection*{Level 2: Application and Analysis}
Level 2 outcomes are those in which the student can apply the material in familiar situations, e.g., can work a problem of familiar structure with minor changes in the details. Upon successful completion of this course, students will be able to:
\begin{enumerate}
\item \textbf{Articulate what counting is and how relevant it is to computer science.}
\item \textbf{Apply the basic principles of counting.}
\item \textbf{Model combinatorial problems using graphs and trees.}
\item \textbf{Describe various types of graphs and their common properties.}
\item \textbf{Identify trees as a fundamental structure in modeling computer science problems.}
\end{enumerate}

\subsection*{Level 1: Knowledge and Comprehension}
Level 1 outcomes are those in which the student has been exposed to the terms and concepts at a basic level and can supply basic definitions. On successful completion of this course, students will be able to:
\begin{enumerate}
\item \textbf{Multiple types of graphs and trees, and how they each are relevant to computer science.}
\end{enumerate}


%\tagstructbegin{tag=Document}
\end{document}